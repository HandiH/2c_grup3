\section{Ainul Filiani}
\subsection{Keterampilan Pemrograman}
\subsubsection{No 1}
\hfill \break
Dibawah ini merupakan penggunaan subplot dan plot bar
\lstinputlisting[firstline=8, lastline=28]{src/6/1174073/praktek/p1174073_bar.py}
Dan dibawah ini merupakan cara pemangilannya
\lstinputlisting[firstline=8, lastline=8]{src/6/1174073/praktek/main_ainulf.py}
\lstinputlisting[firstline=13, lastline=13]{src/6/1174073/praktek/main_ainulf.py}

\subsubsection{No 2}

\hfill \break

Dibawah ini merupakan penggunaan subplot dan plot scatter
\lstinputlisting[firstline=8, lastline=28]{src/6/1174073/praktek/p1174073_scatter.py}
Dan dibawah ini merupakan cara pemangilannya
\lstinputlisting[firstline=9, lastline=9]{src/6/1174073/praktek/main_ainulf.py}
\lstinputlisting[firstline=14, lastline=14]{src/6/1174073/praktek/main_ainulf.py}

\subsubsection{No 3}

\hfill \break

Dibawah ini merupakan penggunaan subplot dan plot pie
\lstinputlisting[firstline=8, lastline=50]{src/6/1174073/praktek/p1174073_pie.py}
Dan dibawah ini merupakan cara pemangilannya
\lstinputlisting[firstline=10, lastline=10]{src/6/1174073/praktek/main_ainulf.py}
\lstinputlisting[firstline=15, lastline=15]{src/6/1174073/praktek/main_ainulf.py}

\subsubsection{No 4}

\hfill \break

Dibawah ini merupakan penggunaan subplot dan plot bar
\lstinputlisting[firstline=8, lastline=28]{src/6/1174073/praktek/p1174073_plot.py}
Dan dibawah ini merupakan cara pemangilannya
\lstinputlisting[firstline=11, lastline=11]{src/6/1174073/praktek/main_ainulf.py}
\lstinputlisting[firstline=16, lastline=16]{src/6/1174073/praktek/main_ainulf.py}

\subsection{Penanganan Error}

\hfill \break

Berikut ini merupakan cara penangganan errornya
\lstinputlisting[firstline=8, lastline=14]{src/6/1174073/praktek/1174073.py}
%%%%%%%%%%%%%%%%%%%%%%%%%%%%%%%%%%%%%%%%%%%%%%%%%%%%%%%%%%%%%%%%%%%%%%%%%%%%%%%%%%%%%%%%%%%%%%%%%%%%%%%%%%%%%%%%%
\section{Sekar Jasmine}
\subsection{Praktek}
\subsubsection{Tugas No 1}
\hfill \break
Dibawah ini merupakan penggunaan subplot dan plot bar
\lstinputlisting[firstline=8, lastline=28]{src/6/1174075/praktek/1174075_bar.py}
Dan dibawah ini merupakan cara pemangilannya
\lstinputlisting[firstline=8, lastline=8]{src/6/1174075/praktek/main_sekar.py}
\lstinputlisting[firstline=13, lastline=13]{src/6/1174075/praktek/main_sekar.py}

\subsubsection{Tugas No 2}

\hfill \break

Dibawah ini merupakan penggunaan subplot dan plot scatter
\lstinputlisting[firstline=8, lastline=28]{src/6/1174075/praktek/1174075_scatter.py}
Dan dibawah ini merupakan cara pemangilannya
\lstinputlisting[firstline=9, lastline=9]{src/6/1174075/praktek/main_sekar.py}
\lstinputlisting[firstline=14, lastline=14]{src/6/1174075/praktek/main_sekar.py}

\subsubsection{Tugas No 3}

\hfill \break

Dibawah ini merupakan penggunaan subplot dan plot pie
\lstinputlisting[firstline=8, lastline=50]{src/6/1174075/praktek/1174075_pie.py}
Dan dibawah ini merupakan cara pemangilannya
\lstinputlisting[firstline=10, lastline=10]{src/6/1174075/praktek/main_sekar.py}
\lstinputlisting[firstline=15, lastline=15]{src/6/1174075/praktek/main_sekar.py}

\subsubsection{Tugas No 4}

\hfill \break

Dibawah ini merupakan penggunaan subplot dan plot bar
\lstinputlisting[firstline=8, lastline=28]{src/6/1174075/praktek/1174075_plot.py}
Dan dibawah ini merupakan cara pemangilannya
\lstinputlisting[firstline=11, lastline=11]{src/6/1174075/praktek/main_sekar.py}
\lstinputlisting[firstline=16, lastline=16]{src/6/1174075/praktek/main_sekar.py}

\subsection{Penanggan Error}

\hfill \break

Berikut ini merupakan cara penangganan errornya
\lstinputlisting[firstline=8, lastline=14]{src/6/1174075/praktek/1174075.py}
%%%%%%%%%%%%%%%%%%%%%%%%%%%%%%%%%%%%%%%%%%%%%%%%%%%%%%%%%%%%%%%%%%%%%%%%%%%%%

\section{Kaka Kamaludin}
\subsection{Soal 1}
\lstinputlisting[firstline=1, lastline=15]{src/6/1174067/Praktek/1174067_bar.py}
\subsection{Soal 2}
\lstinputlisting[firstline=1, lastline=15]{src/6/1174067/Praktek/1174067_scatter.py}
\subsection{Soal 3}
\lstinputlisting[firstline=1, lastline=15]{src/6/1174067/Praktek/1174067_pie.py}
\subsection{Soal 4}
\lstinputlisting[firstline=1, lastline=15]{src/6/1174067/Praktek/1174067_plot.py}
\subsection{keterampilan Penanganan Error}
SyntaxError: invalid token

lagi, salah dalam penulisan " import 1174067\textunderscore bar ", seharusnya menggunakan \textunderscore \textunderscore import\textunderscore \textunderscore('1174067\textunderscore bar')"

%%%%%%%%%%%%%%%%%%%%%%%%%%%%%%%%%%%%%%%%%%%%%%%%%%%%%%%%%%%%%%%%%%%%%%%%%%%%%
\section{Alfadian Owen}
\subsection{Keterampilan Pemrograman}
\subsubsection{No 1}

\hfill \break

buat fungsi library bar 
\lstinputlisting[firstline=8, lastline=23]{src/6/1174091/praktek/1174091_bar.py}
pemanggilannya dengan cara memanggil fungsinya yaitu : bar()

\subsubsection{No 2}

\hfill \break

buat fungsi library scatter
\lstinputlisting[firstline=8, lastline=28]{src/6/1174091/praktek/1174091_scatter.py}
pemanggilannya dengan cara memanggil fungsinya yaitu : scatter()

\subsubsection{No 3}

\hfill \break

buat fungsi library pie
\lstinputlisting[firstline=8, lastline=35]{src/6/1174091/praktek/1174091_pie.py}
pemanggilannya dengan cara memanggil fungsinya yaitu : pie()

\subsubsection{No 4}

\hfill \break

buat fungsi library plot
\lstinputlisting[firstline=8, lastline=32]{src/6/1174091/praktek/1174091_plot.py}
pemanggilannya dengan cara memanggil fungsinya yaitu : plot()

\subsection{Penanganan Error}

\hfill \break

penanganan error
\lstinputlisting[firstline=8, lastline=40]{src/6/1174091/praktek/1174091_error.py}
%%%%%%%%%%%%%%%%%%%%%%%%%%%%%%%%%%%%%%%%%%%%%%%%%%%%%%%%%%%%%%%%%%%%%%%%%%%%%%%%%%%%%%%%%%%%%%%

\section{Fernando Lorencius S}
\subsection{Soal 1}
\lstinputlisting[firstline=8, lastline=30]{src/6/1174072/Praktek/1174072_bar.py}
\subsection{Soal 2}
\lstinputlisting[firstline=9, lastline=28]{src/6/1174072/Praktek/1174072_scatter.py}
\subsection{Soal 3}
\lstinputlisting[firstline=9, lastline=53]{src/6/1174072/Praktek/1174072_pie.py}
\subsection{Soal 4}
\lstinputlisting[firstline=9, lastline=31]{src/6/1174072/Praktek/1174072_plot.py}

\subsection{keterampilan Penanganan Error}
\lstinputlisting[firstline=8, lastline=20]{src/6/1174072/Praktek/1174072_error.py}

%%%%%%%%%%%%%%%%%%%%%%%%%%%%%%%%%%%%%%%%%%%%%%%%%%%%%%%%%%%%%
\section{Alvan Alvanzah/1174077}
\subsection{Ketrampilan Pemrograman}
\begin{enumerate}
    \item Buatlah librari fungsi (file terpisah/library dengan nama NPMbar.py) untuk plot dengan jumlah subplot adalah NPM mod 3 + 2
    \lstinputlisting[firstline=8, lastline=23]{src/6/1174077/praktek/1174077_bar.py}
    
    \item Buatlah librari fungsi (file terpisah/library dengan nama NPMscatter.py) untuk plot dengan jumlah subplot NPM mod 3 + 2
    \lstinputlisting[firstline=8, lastline=23]{src/6/1174077/praktek/1174077_scatter.py}
    
    \item Buatlah librari fungsi (file terpisah/library dengan nama NPMpie.py) untuk plot dengan jumlah subplot NPM mod 3 + 2
    \lstinputlisting[firstline=8, lastline=38]{src/6/1174077/praktek/1174077_pie.py}
    
    \item Buatlah librari fungsi (file terpisah/library dengan nama NPMplot.py) untuk plot dengan jumlah subplot NPM mod 3 + 2
    \lstinputlisting[firstline=8, lastline=23]{src/6/1174077/praktek/1174077_plot.py}
\end{enumerate}

\subsection{Ketrampilan Penanganan Error}
\begin{itemize}
    \item SyntaxError adalah suatu keadaan saat kode python mengalami kesalahan penulisan. Solusinya adalah memperbaiki penulisan kode yang salah.
    
    \item NameError adalah exception yang terjadi saat kode melakukan eksekusi terhadap local name atau global name yang tidak terdefinisi. Solusinya adalah memastikan variabel atau function yang dipanggil ada atau tidak salah ketik.
    
    \item ValueError adalah kesalahan dengan konten objek yang anda tetapkan nilainya. Solusinya adalah memperbaiki nilai dari objek.
    \lstinputlisting[firstline=8, lastline=27]{src/6/1174077/praktek/error.py}
\end{itemize}
%%%%%%%%%%%%%%%%%%%%%%%%%%%%%%%%%%%%%%%%%%%%%%%%%%%%%%%%%%%%%%%%%%%%%%%%%%%%%%%%%%%